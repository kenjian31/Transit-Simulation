\hypertarget{index_intro_sec}{}\section{Introduction\+:}\label{index_intro_sec}
the project is about building transit simulation. The program can be obtain in github \href{https://github.umn.edu/umn-csci-3081-s20/csci3081-shared-upstream/tree/support-code/labs/lab09_project_intro}{\tt https\+://github.\+umn.\+edu/umn-\/csci-\/3081-\/s20/csci3081-\/shared-\/upstream/tree/support-\/code/labs/lab09\+\_\+project\+\_\+intro} using git pull command to obtain the project into local machine, the configure of mechine list below\+: Server\+: Dell Power\+Edge R210 Processor\+: Quad-\/\+Core Intel® Xeon™ E3-\/1220\+V2 Processor @ 3.\+10\+G\+Hz Memory\+: 16\+GB R\+AM with linux system.

the program is execuated by following step\+: Step by step for instructions for C\+SE Labs machines or vole users $<$port\+\_\+number$>$ below should be a number above 8000 that includes the last 3 digits of your student id. So, if your student id number is\+: 1459345, use 8345, or 9345

Navigate to base project directory(project/) Make and run the server\+: \$ cd src \$ make \$ cd .. \$./build/bin/vis\+\_\+sim $<$port\+\_\+number$>$ You must run by doing ./build/bin/vis\+\_\+sim $<$port\+\_\+number$>$. You cannot cd to bin/ and run ./vis\+\_\+sim $<$port\+\_\+number$>$ Run your browser on vole or C\+SE Labs machine, and enter following address in the address bar of your browser (Firefox/\+Chrome)\+: \href{http://127.0.0.1:}{\tt http\+://127.\+0.\+0.\+1\+:}$<$port\+\_\+number$>$/web\+\_\+graphics/project.html Step by step instructions for ssh users \+:

You can S\+SH using a Windows machine with Git Bash.

$\ast$$\ast$$<$port\+\_\+number$>$ below should be a number above 8000 that includes the last 3 digits of your student id.$\ast$$\ast$ {\bfseries So, if your student id number is\+: 1459345, use 8345, or 9345}

ssh -\/L $<$port number$>$=\char`\"{}\char`\"{}$>$\+:127.\+0.\+0.\+1\+:$<$port\+\_\+number$>$ $<$x500$>$$<$cse\+\_\+labs\+\_\+computer$>$.cselabs.\+umn.\+edu Navigate to base project directory(project/) make and start server\+: \$ cd src \$ make \$ cd .. \$./build/bin/vis\+\_\+sim $<$port\+\_\+number$>$ You must run by doing ./build/bin/vis\+\_\+sim $<$port\+\_\+number$>$. You cannot cd to bin/ and run ./vis\+\_\+sim $<$port\+\_\+number$>$ Navigate to the following address into the address bar of a browser (Firefox/\+Chrome) running on your L\+O\+C\+AL machine (e.\+g, your PC)\+: \href{http://127.0.0.1:}{\tt http\+://127.\+0.\+0.\+1\+:}$<$port\+\_\+number$>$/web\+\_\+graphics/project.html

program can be compile by\+: \$ cd src \$ make

the program can be viewed by browser after execuated\+: After running command above, open browser and type in the U\+RL shows above. After browser window opened, prass start batton to start transit simulator

The major improvement in iteration 1 is adding new class bus factory. The main functionality of bus factory is randomly generate 3 kinds of bus which are small, regular, large. To implement the bus factory, I first create smallbus, regularbus and largebus classes that inhirent from \hyperlink{classBus}{Bus} class. There are two way to implement bus factory which are concrete class and abstract class An abstract class is meant to be used as a base class where some or all functions are declared purely virtual and hence can not be instantiated. A concrete class is an ordinary class which has no purely virtual functions and hence can be instantiated. An abstract class is meant to be used as a base class where some or all functions are declared purely virtual and hence can not be instantiated. A concrete class is an ordinary class which has no purely virtual functions and hence can be instantiated. So in this project, I choose to use concrete class. 